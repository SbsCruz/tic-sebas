\documentclass[10pt,twocolumn]{article}
\usepackage{pdfcomment}
\usepackage{amsmath}
\usepackage{amssymb}
\usepackage{graphicx}
\usepackage{caption}
\usepackage{booktabs}

\setlength{\columnsep}{20pt} % Espacio entre columnas

\begin{document}

\title{Testing \LaTeX\ Accessibility}
\author{Zaphod Beeblebrox \\
Sirius Cybernetics Corporation \\
\textit{(Dated: June 3, 2023)}}
\date{}

\maketitle

\section{Introduction}

When GPT-4 first came out, it immediately gained the attention of scientists~\cite{ref1}. Here we test if it can be used to create WCAG-compliant HTML based on \LaTeX\ source code.

\section{Some Useful Formulas}

When formulating what is called Special Relativity today, Einstein started with the Maxwell equations, like this one~\cite{ref3}:
\begin{equation}\pdftooltip{
\oint \vec{E} \cdot d\vec{S} = \frac{1}{\epsilon_0} \iiint edV
}{The surface area integral of a vector is equal to a fraction with 1 in the numerator and the electric constant epsilon 0 in the denominator, and this fraction times the volume integral of charge density.}\end{equation}

Table I lists some of the most commonly used relativistic equations~\cite{ref3}.

And then there is one of the most famous formulas of physics~\cite{ref4}:
\begin{equation}\pdftooltip{
E = \frac{mc^2}{\sqrt{1 - \frac{v^2}{c^2}}}
}{The right-hand side of the equals sign is a fraction. 
Its numerator is mc squared and its denominator has a square root of 1 minus v squared divided by c squared.}\end{equation}

Of course, all of these formulas are merely consequences of the Lorentz transformation between two moving frames of reference. Consider a rotation matrix about the $z$-axis in three-dimensional space,
\begin{equation}\pdftooltip{
x' = Dx
}{The left-hand side of the equation is equal to a fraction. The numerator is D times x, and the denominator is the square root of 1 minus x squared.}\end{equation}

The Lorentz Transformation is a four-dimensional rotation with a matrix $\Lambda^\nu_\mu$:
\begin{equation}\pdftooltip{
x'^\nu = \Lambda^\nu_\mu x^\mu
}{the right-hand side of the equation is a number that is being raised to a power, 
the base of this power is the product of two terms: 
  the first term is a fraction with the numerator equal to ... (a different Greek letter) and the denominator equal to ... (another Greek letter),  
  the second term is just another Greek letter
this entire product has been raised to a certain power.}\end{equation}

where we imply the summation convention,
\begin{equation}\pdftooltip{
x'^\nu = \sum_{\mu=0}^3 \Lambda^\nu_\mu x^\mu
}{A value that is being squared is equal to a sum. The square is the numerator and it consists of four terms. Each term is a product, the first part is called Lambda, it's subscript varies from 0 to 3 and the second part is an x with a superscript that also varies from 0 to 3.}\end{equation}

\section{More Equations}
\begin{equation}\pdftooltip{
    \sqrt{x^2 + 2x - 3}
}{The square root of x squared plus two times x minus three.}\end{equation}
\begin{equation}\pdftooltip{
    x = \frac{\sqrt{19i} + 3}{2}
}{The square root of 19 times i is added to 3. This total is divided by 2.}\end{equation}
\begin{equation}\pdftooltip{
    \cos^2(x + 3) + \sin^2(x)
}{the square of cosine of x plus three is added to the square sine of x.}\end{equation}

\begin{equation}\pdftooltip{
    x = 2 * (\sqrt{25\pi} + i)
}{x equals a product of two terms. 
The first term is a constant that is being multiplied by itself (itself being a certain value). The constant is 2.
The second term has a square root, the thing inside it is 25 times pi, and also includes an imaginary unit i}\end{equation}

\section{Other Remarks}

One could say a lot more about Special Relativity, but don't panic~\cite{ref5}! At the very least, remember equation~(2).

\end{document}
