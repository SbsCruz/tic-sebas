\documentclass[10pt,twocolumn]{article}
\usepackage{amsmath}
\usepackage{amssymb}
\usepackage{graphicx}
\usepackage{caption}
\usepackage{booktabs}

\setlength{\columnsep}{20pt} % Espacio entre columnas

\begin{document}

\title{Testing \LaTeX\ Accessibility}
\author{Zaphod Beeblebrox \\
Sirius Cybernetics Corporation \\
\textit{(Dated: June 3, 2023)}}
\date{}

\maketitle

\section{Introduction}

When GPT-4 first came out, it immediately gained the attention of scientists~\cite{ref1}. Here we test if it can be used to create WCAG-compliant HTML based on \LaTeX\ source code.

\section{Some Useful Formulas}

When formulating what is called Special Relativity today, Einstein started with the Maxwell equations, like this one~\cite{ref3}:
\begin{equation}
\oint \vec{E} \cdot d\vec{S} = \frac{1}{\epsilon_0} \iiint edV
\end{equation}

Table I lists some of the most commonly used relativistic equations~\cite{ref3}.

And then there is one of the most famous formulas of physics~\cite{ref4}:
\begin{equation}
E = \frac{mc^2}{\sqrt{1 - \frac{v^2}{c^2}}}
\end{equation}

Of course, all of these formulas are merely consequences of the Lorentz transformation between two moving frames of reference. Consider a rotation matrix about the $z$-axis in three-dimensional space,
\begin{equation}
x' = Dx
\end{equation}

The Lorentz Transformation is a four-dimensional rotation with a matrix $\Lambda^\nu_\mu$:
\begin{equation}
x'^\nu = \Lambda^\nu_\mu x^\mu
\end{equation}

where we imply the summation convention,
\begin{equation}
x'^\nu = \sum_{\mu=0}^3 \Lambda^\nu_\mu x^\mu
\end{equation}

\section{More Equations}
\begin{equation}
    \sqrt{x^2 + 2x - 3}
\end{equation}
\begin{equation}
    x = \frac{\sqrt{19i} + 3}{2}
\end{equation}
\begin{equation}
    \cos^2(x + 3) + \sin^2(x)
\end{equation}

\begin{equation}
    x = 2 * (\sqrt{25\pi} + i)
\end{equation}

\section{Other Remarks}

One could say a lot more about Special Relativity, but don't panic~\cite{ref5}! At the very least, remember equation~(2).

\end{document}
